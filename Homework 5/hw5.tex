%%%%%%%%%% DOCUMENT STUFF %%%%%%%%%%

\documentclass[10pt,letterpaper]{article}
\usepackage{mathtools}
\usepackage{amsmath}
\usepackage{amssymb}
\usepackage{datetime}
\usepackage{setspace}
\usepackage{tkz-graph}
\usepackage[margin=1in]{geometry}
\usepackage{courier}
\usepackage{listings}
\usepackage{mips}
\usepackage{graphicx}
\usepackage{graphics}
\usepackage{enumitem}
\usepackage{colortbl}

%%%%%%%%%% FORMATTING %%%%%%%%%%

\newdate{date}{8}{05}{2017}
\spacing{1.5}
\date{\displaydate{date}}
\setcounter{secnumdepth}{0}
\newcommand\tab[1][0.5cm]{\hspace*{#1}}
\newcommand*\circled[1]{\tikz[baseline=(char.base)]{
            \node[shape=circle,draw,inner sep=2pt] (char) {#1};}}
\lstset{language=[mips]Assembler,basicstyle=\ttfamily}
\graphicspath{{images/}}

%%%%%%%%%% CONTENT %%%%%%%%%%

%%%%% COVER PAGE %%%%%

\begin{document}
\title{CS M151B: Homework 5}
\author{
	Jonathan Woong\\
	804205763\\
	Spring 2017\\
	Discussion 1A}
\maketitle
\pagebreak

%%%%%%%%%%

\begin{enumerate}[label=\textbf{Problem \arabic*.}]
\item Consider the pipelined MIPS implementation shown in Figure 4.51 (page 304). The value of the \texttt{PCSrc} signal is computed in the MEM stage. Could the \texttt{PCSrc} signal be computed in the EX stage instead? If so, what would be the advantages of making this change? Also, what would be the disadvantages of making this change?\\
Yes, the \texttt{PCSrc} signal can be computed in the EX stage. This will give the advantage of reducing branch stall to one cycle. The disadvantage is that additional forwarding and hazard detection is required, and we must also implement the ability to flush instructions in the IF stage.
\item Consider the pipelined MIPS implementation shown in Figure 4.51 (page 304). Consider each of the following two instructions separately. For each of these instructions (separately), what is the value of the \texttt{PCSrc} signal when the instruction is in the MEM pipeline stage? Explain your answer.
	\begin{enumerate}[label=\alph*.]
	\item \texttt{add $\tab$ \$5,\$15,\$24}\\
	The value of the \texttt{PCSrc} signal at the MEM stage is 0 because the Branch signal for R-format instructions is set to 0, meaning the the next sequential PC address will be PC + 4.
	\item \texttt{beq $\tab$ \$1,\$1,200}\\
	The value of the \texttt{PCSrc} signal at the MEM stage is 1 because the Branch signal is always asserted for \texttt{beq} instructions, and the Zero signal from the ALU will be 1 due to the fact that register \$1 is always equal to itself. This will set the PC to be the result of the ALU address computation.
	\end{enumerate}
\item Consider the following code on the pipelined datapath of Figure 4.60 on page 316:
\begin{lstlisting}
add 	$2, $3, $1
sub 	$4, $3, $5
add 	$5, $3, $7
add 	$7, $6, $1
add 	$8, $2, $6
\end{lstlisting}
During the fifth cycle, which registers of the register file are read and which register of the register file is written?
\begin{center}
\scalebox{0.75}{
\begin{tabular} { |c|c|c|c|c|c| }
\hline
Instruction & CC 1 & CC 2 & CC 3 & CC 4 & CC 5 \\
\hline
\texttt{add} & IF & ID & EX & MEM & WB \\
\hline
\texttt{sub} & & IF & ID & EX & MEM \\
\hline
\texttt{add} & & & IF & ID & EX \\ 
\hline
\texttt{add} & & & & IF & ID \\
\hline 
\texttt{add} & & & & & IF \\
\hline
\end{tabular}
}
\end{center}
\boxed{Read: \$6, \$1}\\
\boxed{Written: \$2}
\item Consider the pipelined MIPS implementation shown in Figure 4.51 (page 304). Figure 4.51 includes support for the \texttt{beq} instruction. However, Figure 4.51 ignores the existence of control dependencies. Specifically, Figure 4.51 does not include the required control hazard detection and stall implementation. Your task is to add control hazard detection and stall implementation to Figure 4.51. Do not use any branch prediction mechanism. Do not modify the basic datapath (you cannot use the modified datapath shown in Figure 4.62). As a starting point, consider the hazard detection and stall implementation shown in Figure 4.60 (page 316), with the hazard detection unit logic described on pages 313-314. Modify this circuit as necessary and add the possibly modified circuit to Figure 4.51. Using the same style as in page 314, specify the logic of the hazard detection unit.\\
Hazard detection unit: \\
\texttt{Case 1: if (IF/ID.Opcode=000100 and \\
\tab (ID/EX.RegisterRd = IF/ID.RegisterRs or ID/EX.RegisterRd = IF/ID.RegsiterRt) or (ID/EX.RegisterRt = IF/ID.RegisterRs or ID/EX.RegisterRt = IF/ID.RegsiterRt)) \\
\tab stall 3 cycles \\ 
Case 2: if (IF/ID.Opcode=000100 and \\
\tab (EX/MEM.RegisterRd = IF/ID.RegisterRs or EX/MEM.RegisterRd = IF/ID.RegsiterRt)) \\
\tab stall 2 cycles \\
Case 3: if (IF/ID.Opcode=000100 and \\
\tab MEM/WB.RegWrite=1 and (MEM/WB.RegisterRd = IF/ID.RegisterRs or MEM/WB.RegisterRd = IF/ID.RegsiterRt)) \\
\tab stall 1 cycle \\
Case 4: if (MEM/WB.Branch and MEM/WB.Zero)\\
\tab flush IF/ID, ID/EX, EX/MEM} \\
\textbf{Case 1,2,3}: The first line identifies the \texttt{beq} instruction, since it is the only one to use the opcode 000100. The second line identifies that either the Rs or Rt of \texttt{beq} intends to be written by a previous instruction (either \texttt{lw} or an R-type instruction). The stall is accomplished by the MUX that is controlled by the Hazard Detection Unit signal, which sets all control signals to 0. After stalling the correct number of cycles, the correct values for the Rs and Rt registers are written by the previous instruction.\\
\textbf{Case 4}: The line identifies the condition that the branch must be taken. In this case, we flush the IF/ID, ID/EX, and EX/MEM register values to ensure that instructions that execute after branching are not affected by instructions that would have executed if the branch was not taken. This can be accomplished by setting all control signals to 0 (using the MUX controlled by Hazard Detection Unit signal) and asserting IF.Flush for three cycles.\\
\begin{center}
\scalebox{0.75}{\includegraphics{images/hw5-4.png}}
\end{center}
\item Figure 4.65 (page 325) shows the pipelined MIPS implementation with the details of how stalls due to the \texttt{lw} and \texttt{beq} instructions are implemented. As discussed in class, the book doesn't fully explain how the stalls for \texttt{beq} are implemented. Assume that the implementation is as shown on slide 7.54 in the class notes. As shown in Figure 4.65, the IF/ID register is a selectively-modified register. Assume that this register is implemented with flip-flops, as shown on the right side of slide 5.19 in the class notes.
	\begin{enumerate}[label=\Alph*)]
	\item On a copy of Figure 4.65, show the digital circuit (gate-level implementation) that generates the IF.Flush signal. Be sure that it is clear exactly how it is connected to the rest of the implementation. If it is connected to a module that generates multiple outputs, clearly identify the specific output to which your circuit is connected. 
	\begin{center}
	\scalebox{0.75}{\includegraphics{images/hw5-5-1.png}}
	\end{center}
	\item Given the assumption above regarding how the IF/ID register is implemented, show the digital circuit (gate-level implementation) that uses the IF.Flush signal to implement the desired functionality. Note that this part should be shown separately from the copy of Figure 4.65.
	\hspace*{-1.5cm}\includegraphics[width=\textwidth]{images/hw5-5-2.png}
	\end{enumerate}
\item Refer to the following fragment of MIPS code:
\begin{lstlisting}
sw 	r16,12(r6)
lw 	r16,8(r6)
beq 	r5,r4,Label 	# Assume r5!=r4
add 	r5,r1,r4
slt 	r5,r15,r4
\end{lstlisting}
Assume that individual pipeline stages have the following latencies:
\begin{center}\begin{tabular} { |c|c|c|c|c| }
\hline
IF & ID & EX & MEM & WB \\
\hline
200 ps & 120 ps & 150 ps & 190 ps & 100 ps \\
\hline
\end{tabular}\end{center}
Assume that all branches are perfectly predicted (this eliminates all control hazards) and that no delay slots are used. If we change load/store instructions to use a register (without an offset) as the address, these instructions no longer need to use the ALU. As a result, MEM and EX stages can be overlapped and the pipeline has only 4 stages. Change this code to accomodate this changed ISA. Assuming this change does not affect clock cycle time, what speedup is achieved in this instruction sequence?
\pagebreak
\begin{center}
Before change:\\
\hspace*{-1.5cm}\begin{tabular}[width=\pagewidth] { |c|c|c|c|c|c|c|c|c|c|c|c|c|c| }
\hline
Instruction & CC 1 & CC 2 & CC 3 & CC 4 & CC 5 & CC 6 & CC 7 & CC 8 & CC 9 & CC 10 & CC 11 & Total\\
\hline
\texttt{sw} & \cellcolor{blue!25}IF & ID & EX & \cellcolor{blue!25}MEM & WB & & & & & & & \\
\hline
\texttt{lw} & & \cellcolor{blue!25}IF & ID & EX & \cellcolor{blue!25}MEM & WB & & & & & & \\
\hline
\texttt{beq} & & & \cellcolor{blue!25}IF & ID & EX & MEM & WB & & & & & \\
\hline
\texttt{add} & & & & & & \cellcolor{blue!25}IF & ID & \cellcolor{blue!25}EX & \cellcolor{blue!25}MEM & WB & & \\
\hline
\texttt{slt} & & & & & & & \cellcolor{blue!25}IF & ID & EX & \cellcolor{blue!25}MEM & \cellcolor{blue!25}WB & \\ 
\hline
\textbf{time} & 200 & 200 & 200 & 190 & 190 & 200 & 200 & 150 & 190 & 190 & 100 & 2010\\
\hline
\end{tabular}\\
After change:\\
\hspace*{-1.5cm}\begin{tabular} { |c|c|c|c|c|c|c|c|c|c|c|c|c| }
\hline
Instruction & CC 1 & CC 2 & CC 3 & CC 4 & CC 5 & CC 6 & CC 7 & CC 8 & CC 9& Total\\
\hline
\texttt{sw} & \cellcolor{blue!25}IF & ID & EX/MEM & WB & & & & & & \\
\hline
\texttt{lw} & & \cellcolor{blue!25}IF & ID & \cellcolor{blue!25}EX/MEM & WB & & & & & \\
\hline
\texttt{beq} & & & \cellcolor{blue!25}IF & ID & EX/MEM & WB & & & & \\
\hline
\texttt{add} & & & & & \cellcolor{blue!25}IF & ID & \cellcolor{blue!25}EX/MEM & WB & & \\
\hline
\texttt{slt} & & & & & & \cellcolor{blue!25}IF & ID & \cellcolor{blue!25}EX/MEM & \cellcolor{blue!25}WB & \\ 
\hline
\textbf{time} & 200 & 200 & 200 & 190 & 200 & 200 & 190 & 190 & 100 & 1670\\ 
\hline
\end{tabular}
\end{center}
\begin{lstlisting}
sw 	r16,r6
lw 	r16,r6
beq 	r5,r4,Label 	# Assume r5!=r4
add 	r5,r1,r4
slt 	r5,r15,r4
\end{lstlisting}
Total speedup $= 2010 - 1670 = \boxed{340 \ ps} = 1 - \frac{1670}{2010} = 0.1691 = \boxed{16.91\%}$
\item Assuming stall-on-branch and no delay slots, what speedup is achieved on this code if branch outcomes are determined in the ID stage, relative to the execution where branch outcomes are determined in the EX stage?
\begin{center}
After change:\\
\hspace*{-1.5cm}\begin{tabular}[width=\pagewidth] { |c|c|c|c|c|c|c|c|c|c|c|c|c|c| }
\hline
Instruction & CC 1 & CC 2 & CC 3 & CC 4 & CC 5 & CC 6 & CC 7 & CC 8 & CC 9 & CC 10 & Total\\
\hline
\texttt{sw} & \cellcolor{blue!25}IF & ID & EX & \cellcolor{blue!25}MEM & WB & & & & & & \\
\hline
\texttt{lw} & & \cellcolor{blue!25}IF & ID & EX & MEM & WB & & & & & \\
\hline
\texttt{beq} & & & \cellcolor{blue!25}IF & ID & EX & MEM & WB & & & & \\
\hline
\texttt{add} & & & & & \cellcolor{blue!25}IF & ID & \cellcolor{blue!25}EX & \cellcolor{blue!25}MEM & WB & & \\
\hline
\texttt{slt} & & & & & & \cellcolor{blue!25}IF & ID & EX & \cellcolor{blue!25}MEM & \cellcolor{blue!25}WB & \\ 
\hline
\textbf{time} & 200 & 200 & 200 & 190 & 200 & 200 & 150 & 190 & 190 & 100 & 1820\\
\hline
\end{tabular}\end{center}
Total speedup $= 2010 - 1820 = \boxed{190 \ ps} = 1 - \frac{1820}{2010} = 0.0945 = \boxed{9.45\%}$
\item An ISA that is very similar to the MIPS ISA supports 95 opcodes and 64 registers. Instructions are 32 bits wide. There is a class of instructions in this ISA that specify two register arguments and one signed (2's complement) immediate. What is the maximum possible range of values of this immediate?
\begin{center}\begin{tabular} { |c|c|c|c| }
\hline
opcode & Rs & Rt & Immediate \\
\hline
7 bits & 6 bits & 6 bits & 13 bits \\
\hline
\end{tabular}\end{center}
\boxed{\text{Range = } (-2^{12},2^{12}-1)=(-4096,4095)} 
\item Implement the following C code in MIPS using indexed addressing and in PowerPC using update addressing:
\begin{center}
\texttt{for (i=0; i != 10; i++) \{ a[2i] = b[i] + i; \}}
\end{center}
Assume that the base address at \texttt{a[]} is stored in \texttt{\$a0 (\$4)}, and the base address of \texttt{b[]} is stored in \texttt{\$a1 (\$5)}.
\begin{lstlisting}[escapeinside={(*}{*)}]
MIPS: 	addi 	$1,$0,0 	# $1 = i (*$\leftarrow$*) 0
 	addi 	$6,$1,40 	# $6 = $a2 (*$\leftarrow$*) 40
loop: 	beq 	$1,$6,end 	# if (i == 10): exit loop	
	lw 	$9,$5+$1 	# $9 = $t1 (*$\leftarrow$*) M[$a1 + i] = b[i]
	add 	$9,$9,$1 	# $t1 (*$\leftarrow$*) b[i] + i
	sll 	$7,$1,1 	# $7 = $a3 (*$\leftarrow$*) 2i
	sw 	$9,$4+$7 	# M[$a0 + 2i] (*$\leftarrow$*) $t1 (*$\Leftrightarrow$*) a[2i] (*$\leftarrow$*) b[i] + i
	addi 	$1,$1,4 	# i (*$\leftarrow$*) i + 1
	j 	loop 		# repeat loop
end: 	

PowerPC:addi 	$1,$0,0 	# $1 = i (*$\leftarrow$*) 0
 	addi 	$6,$1,40 	# $6 = $a2 (*$\leftarrow$*) 40
	addi 	$8,$5,$1 	# $8 = $t0 (*$\leftarrow$*) $a1 + i
	lw 	$8,0($8) 	# $t0 (*$\leftarrow$*) M[$a1 + i] = b[i]
loop:	beq 	$1,$6,end 	# if (i == 10): exit loop
	add 	$9,$8,$1 	# $t1 (*$\leftarrow$*) b[i] + i
	sll 	$10,$1,1 	# $10 = $t2 (*$\leftarrow$*) 2i
	add 	$7,$4,$10 	# $7 = $a3 (*$\leftarrow$*) $a0 + 2i 
	sw 	$9,0($7) 	# M[$a0 + 2i] = a[2i] (*$\leftarrow$*) b[i] + i
	(*\textbf{lwu}*)     $8,4($1) 	     # $8 (*$\leftarrow$*) M[$8 + 4]; i (*$\leftarrow$*) i + 4
	j 	loop
end: 
\end{lstlisting}
\end{enumerate}
\end{document}
%%%%%%%%%% DOCUMENT STUFF %%%%%%%%%%

\documentclass[10pt,letterpaper]{article}
\usepackage{mathtools}
\usepackage{amsmath}
\usepackage{amssymb}
\usepackage{datetime}
\usepackage{setspace}
\usepackage{tkz-graph}
\usepackage[margin=1in]{geometry}
\usepackage{courier}
\usepackage{listings}
\usepackage{mips}
\usepackage{graphicx}
\usepackage{graphics}
\usepackage{enumitem}
\usepackage{colortbl}

%%%%%%%%%% FORMATTING %%%%%%%%%%

\newdate{date}{22}{05}{2017}
\spacing{1.5}
\date{\displaydate{date}}
\setcounter{secnumdepth}{0}
\newcommand\tab[1][0.5cm]{\hspace*{#1}}
\newcommand*\circled[1]{\tikz[baseline=(char.base)]{
            \node[shape=circle,draw,inner sep=2pt] (char) {#1};}}
\lstset{language=[mips]Assembler,basicstyle=\ttfamily}
\graphicspath{{images/}}

%%%%%%%%%% CONTENT %%%%%%%%%%

%%%%% COVER PAGE %%%%%

\begin{document}
\title{CS M151B: Homework 7}
\author{
	Jonathan Woong\\
	804205763\\
	Spring 2017\\
	Discussion 1A}
\maketitle
\pagebreak

%%%%%%%%%%

\begin{enumerate}[label=\textbf{Problem \arabic*.}]
\item In the system described in Ed2-Chap8 pp. 665-666, the memory system takes 200 ns to read the first four words, and each additional four words require 20 ns. The memory system is modified so that it takes 130 ns to read the first four words and 12 ns to read each additional four words, find the sustained bandwidth and the latency for a read of 256 words for transfers that use 4-word blocks and for transfers that use 16-word blocks. Also compute the effective number of bus transactions per second for each case.\\
For 4-word block transfers, each block takes
	\begin{itemize}
	\item 1 clock cycle to send address to memory
	\item $\frac{130 \ ns}{5\frac{ns}{cycle}} = 26$ clock cycles to read memory 
	\item 2 clock cycles to send data from memory
	\item 2 idle clock cycles between this transfer and next
	\end{itemize}
This is a total of 31 cycles, and $\frac{256}{4}=64$ transactions are needed, so the entire transfer takes $31\times64=1984$ clock cycles. Thus the latency is $1984 \ cycles \times 5\frac{ns}{cycle} = 9920$ ns. The number of bus transactions per second is 
\[\boxed{64 \text{ transactions} \times \frac{1 \ second}{9920 \ ns} = 6.4516 \text{M transactions/second}}\]
The bus bandwidth is 
\[\boxed{(256 \times 4) \text{bytes} \times \frac{1 \ second}{9920 \ ns} = 103.2258 \text{MB/sec}}\]
For 16-word block transfers, the first block requires
	\begin{itemize}
	\item 1 clock cycle to send address to memory
	\item 130 ns or 26 cycles to read first four words in memory 
	\item 2 clock cycles to send data of block, during which time the read of four words in next block is started
	\item 2 idle clock cycles between transfers, during which the read of next block is completed
	\end{itemize}
Each of the three remaining 4-word blocks requires repeating only the last two steps.
Thus, the total number of cycles for each 16-word block is $1+26+4\times(2+2)=43$ cycles, and $\frac{256}{16}=16$ transactions are needed, so the entire transfer takes $43\times16=688$ cycles. Thus the latency is $688 \ cycles \times 5 \frac{ns}{cycle} = 3440$ ns, which is roughly one-third of the latency for the case with 4-word blocks. The number of bus transactions per second with 16-word blocks is
\[\boxed{16 \text{ transactions} \times \frac{1 \ second}{3440 \ ns} = 4.6512 \text{M transactions/second}}\]
The bus bandwidth with 16-word blocks is
\[\boxed{(256\times4)\text{bytes}\times\frac{1 \ second}{3440 \ ns}=297.6744 \text{MB/second}}\] which is 2.88 times higher than the 4-word blocks.
\item Is there any situation where interrupt-driven I/O is prefereable to DMA-based I/O? Explain.\\
Yes, in situations where the CPU expects input from a device that is unpredictable (such as human typing on the keyboard), interrupt-driven I/O is preferable to DMA-based I/O. This is because the amount of time it takes to handle the interrupt signal of human typing requires immediate action from the CPU, and cannot be improved in response time if the CPU delegates the memory access to the DMA.
\item Design a ``fair'' bus arbiter. There are four devices connected to the bus. The arbiter will have four ``$bus \ request$'' lines as inputs. The outputs of the arbiter are four ``$bus \ grant$'' lines. When a device wants the bus it asserts its bus request line. A device gets the bus when its bus grant line is asserted by the arbiter. A device that gets the bus uses it for one cycle. The arbitration for any particular bus cycle occurs during the previous cycle. A device that is currently using the bus $can$ contend for use of the bus during the next cycle. At most one device can use the bus during one cycle. Any number of devices (zero to four) can contend for the bus during every cycle. The arbiter enforces ``fairness'' among the devices by dynamically changing the relative priority of the devices. Denote the devices: $D_0,D_1,D_2$ and $D_3$. When the system is initialized, $D_0$ has top priority, then $D_1$ and $D_2$ with $D_3$ having the lowest priority. Note that even the lowest priority device may get the bus if none of the other devices request it. When some device $D_i (0\leq i \leq 3)$ gets the bus, the top priority is changed to device $D_j$ with $j=(i+1)mod4$. The priority of devices is then $D_{(i+1)mod4},D_{(i+2)mod4},D_{(i+3)mod4}$, and $D_{(i+4)mod4}$.
	\begin{enumerate}[label=\Alph*)]
	\item Design the arbiter. Your design should be at the level of gates and flip-flops. Show your work (truth tables, etc.).\\
	\hspace*{-2cm}\includegraphics[width=\textwidth]{images/hw7-3.png}\\
	Combine each individual bus request signal $(R_0,R_1,R_2,R_3)$ into a signal $R$, which is a 4-bit wide line with the left-most bit representing $R_0$ and the right-most bit representing $R_3$. Let $R$ be one input to four AND gates ($A_0, A_1, A_2, A_3$), with the other input to this AND gate being the bit map representing the priority $P = (P_0,P_1,P_2,P_3)$ where $P_0$ is top priority and $P_3$ is lowest priority. Initially, $P_0$ is 1000 ($D_0$), $P_1$ is 0100 ($D_1$), $P_2$ is 0010 ($D_2$), and $P_3$ is 0001 ($D_3$). The output of $A_0$ will be 0000 if $R_0$ is 0, otherwise the output is $P_0$ (this applies to $A_0,A_1,A_2,A_3$). The XNOR gates are used to check whether an output of an AND gate is 0000, they take as input the output of an AND gate and the value 0000. For example, if the output of $A_0$ equals 0000, then the output of $X_0$ equals 1 (this is true for $X_0,X_1,X_2$). The output of these XNOR gates is the input to a MUX, which selects the highest priority non-zero bitmap to output as the grant signal $G$. $G$ is then the input to a shifter which shifts $G$ to the left by 1 (assume wrapping). The table below shows the execution of two cycles assuming $R = (0,0,1,1),(1,1,0,0)$ :
	\[\hspace*{-2cm}\begin{tabular}[width=\textwidth] { |c|c|c|c|c|c|c|c|c|c|c|c|c|c|c| }
	\hline
	$R$ & $P_0$ & $P_1$ & $P_2$ & $P_3$ & $A_0$ & $A_1$ & $A_2$ & $A_3$ & $X_0$ & $X_1$ & $X_2$ & $A_4$ & $A_5$ & $G$ \\  
	\hline
	0011 & 1000 & 0100 & \cellcolor{blue!25}0010 & 0001 & 0000 & 0000 & \cellcolor{blue!25}0010 & 0001 & 1 & 1 & 0 & 1 & 0 & \cellcolor{blue!25}0010 \\
	\hline
	1100 & \cellcolor{blue!25}0100 & 1000 & 0001 & 0010 & 1000 & \cellcolor{blue!25}0100 & 0000 & 0000 & 0 & 0 & 1 & 0 & 0 & \cellcolor{blue!25}0100 \\
	\hline 
	\end{tabular}\]
	\item Show a timing diagram that includes the clock, the bus request lines, the bus grant lines, and the bus data lines for five cycles under the following conditions: 
		\begin{itemize}
			\item The system is intiailized before the first cycle. 
			\item During the first cycle $D_1$ and $D_3$ request the bus.
			\item During the second cycle $D_2$ requests the bus.
		\end{itemize}
		\[\begin{tabular} { |c|c|c|c| }
		\hline
		Clock & $R_3R_2R_1R_0$ & $G_3G_2G_1G_0$ & Data \\
		\hline
		1 & 1010 & 0000 & \\
		\hline
		2 & 0100 & 0010 & \\
		\hline
		3 & 0000 & 0000 & $D_1$ \\
		\hline
		4 & 0000 & 0100 & \\
		\hline
		5 & 0000 & 0000 & $D_2$ \\
		\hline
		\end{tabular}\]
	\end{enumerate} 
\item Recall that the MIPS ISA is big endian. Consider the pipelined MIPS implementation shown in Figure 4.60, page 316, with the enhancement for supporting \texttt{lw} and \texttt{sw} shown in Figure 4.57, page 312. In this implementation, the box labeled ``Data memory'' is a $2^{30}\times32$ memory -- a 32-bit word can be either read or written to the memory in every access. Hence, the address input to this memory consists of just 30 bits -- the 30 most-significant bits of the effective address computed by the ALU. Your task is to add to this implementation support for the \texttt(sb) (store byte) instruction, documented on page A-68 in the book and page 2.22 in the class notes. You \textbf{cannot} modify the implementation of the data memory in \textbf{any way}. Be sure to consider any iterations between the new \texttt{sb} instruction and the already supported instructions that must continue to work.
	\begin{enumerate}[label=\Alph*)]
	\item Explain your modifications in 3-5 clear sentences.
		\begin{itemize}
			\item Add a new control signal \texttt{Byte} that is an input to a MUX. 
			\item Add a MUX with the first input as the 32-bit word read from memory. The second input is the 8 LSB of the 32-bit word read from memory and with 24 bits of 0 appended to the MSB. When \texttt{Byte} is asserted, the second input becomes the output.
			\item The output of the \texttt{Byte} MUX is input to the MUX controlled by the \texttt{MemtoReg} signal. 
		\end{itemize}
	\item On a copy of Figure 4.60, show the modifications.\\
	\hspace*{-2cm}\includegraphics[width=\textwidth]{images/hw7-4.png}
	\item List all the required new control signals with an explanation and identify with on the datapath figure.\\
	Look at part A and B. 
	\item Changes are required to the main $Control$ circuit. Show those changes using a table similar to the onw shown on slide 7.30 in the class notes.\[
	\hspace*{-3.25cm}\begin{tabular} { |c|c|c|c|c|c|c|c|c|c| }
	\hline
	\textbf{Instruction} & \textbf{RegDst} & \textbf{AluOp} & \textbf{ALUSrc} & \textbf{Branch} & \textbf{MemRead} & \textbf{MemWrite} & \textbf{RegWrite} & \textbf{MemtoReg} & \textbf{Byte} \\
	\hline
	R-format & 1 & 10 & 0 & 0 & 0 & 0 & 1 & 0 & 0 \\
	\hline
	\texttt{lw} & 0 & 00 & 1 & 0 & 1 & 0 & 1 & 1 & 0\\
	\hline
	\texttt{sw} & X & 00 & 1 & 0 & 0 & 1 & 0 & X & 0\\
	\hline
	beq & X & 01 & 0 & 1 & 0 & 0 & 0 & X & X\\
	\hline
	\texttt{sb} & X & 00 & 1 & 0 & 0 & 1 & 0 & X & 1\\
	\hline
	\end{tabular}
	\]
	\end{enumerate}
\item Consider a 16$M \times 1$ DRAM chip. The notation used below is: \texttt{read 100} is a read operation that reads one bit from address 100, \texttt{write 350} is a write operation that writes one bit from address 350. One of two possible sequences of operations are performed:
	\begin{enumerate}[label=\Alph*)]
	\item \texttt{read 50, read 500, read 5000}
	\item \texttt{write 50, write 500, write 500}
	\end{enumerate}
What sequence of operations (A or B) can be completed faster? Explain your answer.\\
A $16M\times 1$ chip is implemented with a $4096 \times 4096$ array. 
This means \texttt{read 50} will cost 1 row access and 1 column access. Since \texttt{read 500} is in the same row as \texttt{read 50}, only 1 additional column access is required. \texttt{read 5000} costs 1 row access and 1 column access. The total number of accesses required by A is 5.\\
\texttt{write 50}, \texttt{write 500}, \texttt{write 500} are all in the same row, so the cost is 1 row access and 3 column accesses. The total number of accesses required by B is 4.\\
$\therefore$ B will complete faster.
\item You have $4M\times8$ DRAM chips (as many as you want). You must use these chips to build the data memory for the single-cycle MIPS processor described in chapter 4 of the textbook. What is the minimum size memory you can build? Specify the size in bits. Explain your answer.\\
A $4M\times8$ chip is implemented with a $2048\times2048$ array. Each row access will access 2K bits and throw away all but 8 bits. Since we can access 8 bits at a time, we will need a MUX that takes $\frac{2048}{8}=256$ inputs. This means that the length of the address that acts as input to this MUX is 8 bits.\\
We can combine 4 of the $4M\times8$ DRAM chips to handle accessing a 32-bit memory address. The first DRAM chip will take as input bits 0-7 of the address, the second DRAM chip will take as input bits 8-15, the third chip will take input bits 16-23, the last chip will take input bits 24-31. The 8-bit output of each DRAM chip are concatenated to represent the 32-bit word that is accessed.\\
The minimum size memory you can build is $4(2048\times2048)=16M$ bits.
\end{enumerate}
\end{document}
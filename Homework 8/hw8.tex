%%%%%%%%%% DOCUMENT STUFF %%%%%%%%%%

\documentclass[10pt,letterpaper]{article}
\usepackage{mathtools}
\usepackage{amsmath}
\usepackage{amssymb}
\usepackage{datetime}
\usepackage{setspace}
\usepackage{tkz-graph}
\usepackage[margin=1in]{geometry}
\usepackage{courier}
\usepackage{listings}
\usepackage{mips}
\usepackage{graphicx}
\usepackage{graphics}
\usepackage{enumitem}
\usepackage{colortbl}

%%%%%%%%%% FORMATTING %%%%%%%%%%

\newdate{date}{31}{05}{2017}
\spacing{1.5}
\date{\displaydate{date}}
\setcounter{secnumdepth}{0}
\newcommand\tab[1][0.5cm]{\hspace*{#1}}
\newcommand*\circled[1]{\tikz[baseline=(char.base)]{
            \node[shape=circle,draw,inner sep=2pt] (char) {#1};}}
\lstset{language=[mips]Assembler,basicstyle=\ttfamily}
\graphicspath{{images/}}

%%%%%%%%%% CONTENT %%%%%%%%%%

%%%%% COVER PAGE %%%%%

\begin{document}
\title{CS M151B: Homework 8}
\author{
	Jonathan Woong\\
	804205763\\
	Spring 2017\\
	Discussion 1A}
\maketitle
\pagebreak

%%%%%%%%%%

\begin{enumerate}[label=\textbf{Problem \arabic*.}]
\item Compute the effective access time, in seconds, of a memory system consisting of a cache and main memory with a processor running at 2 GHz. The cache hit time is one cycle. The miss penalty is 45 cycles. The hit rate is 93\%.
\[\boxed{\text{AMAT = time}_{\text{hit}} + \text{ miss rate } \times \text{ miss penalty} = 1 + (1-0.93) \times 45 = 4.15 \text{ clock cycles}}\]
\item Compute the \textbf{total} number of bits of storage required to implement a cache that is similar to the cache shown in Figure 5.18 on page 408, except that the associativity of the cache is 16 instead of 4. All the other parameters (cache size, block size) are the same as in Figure 5.18. Note that the number you are being asked to compute is different from the ``size of the cache,'' which refers only to the number of bits of data stored in the cache.\\
1024 blocks per set\\
2 sets\\
6 bit index\\
24 bit tag\\
1 bit verify\\
32 bit data
\[\text{total \# of bits} = (\text{verify + tag + data}) \times (\text{\# sets}) \times (\text{\# blocks per set}) = (1 + 24 + 32) \times 2 \times 1024 = \boxed{116736 \text{ bits}}\]
\item A CPU generates 50 bit addresses. The memory system is word-addressable. The cache contains 8192 block frames. The tag size is 34 bits. Based on this information, what is the block size? If a range of block sizes is possible, specify this range.\\
Direct-mapped:\\
$\log_2(8192) = 13$ bit index\\
$34 + 13 = 47$ bits for tag and index\\
$50-47 = 3 \text{ bits}\Rightarrow 2^3 =$ 8 \text{ words per block}
Fully associative:\\
$2^16$ words\\
Range:\boxed{(8,65536)\text{ words}}
\item 5.2.2: For each of these references, identify the binary address, the tag, and the index given a direct-mapped chace with two-word blocks and a total size of 8 blocks. Also list if each reference is a hit or a miss, assuming the cache is initially empty. In addition, show the final contents of the cache. This means that, for each block frame that contains a valid block, show the block number, the word address of the words in each block, and the tag of that block.
	\[\begin{tabular} {|c|c|c|c|c|}
	\hline
	\textbf{Address} & \textbf{Binary Address} & \textbf{Tag} & \textbf{Index} & \textbf{Hit/Miss} \\
	\hline
	$3_{10}$ & $00000011_2$ & 00000 & 001 & M \\
	\hline
	$180_{10}$ & $10110100_2$ & 10110 & 010 & M \\
	\hline
	$43_{10}$ & $00101011_2$ & 00101 & 010 & M \\
	\hline
	$2_{10}$ & $00000010_2$ & 00000 & 010 & M \\
	\hline
	$191_{10}$ & $10111111_2$ & 10111 & 111 & M \\
	\hline
	$88_{10}$ & $01011000_2$ & 01011 & 100 & M \\
	\hline
	$190_{10}$ & $10111110_2$ & 10111 & 111 & H \\
	\hline
	$14_{10}$ & $00001110_2$ & 00001 & 111 & M \\
	\hline
	$181_{10}$ & $10110101_2$ & 10110 & 010 & H \\
	\hline
	$44_{10}$ & $00101100_2$ & 00101 & 110 & H \\
	\hline
	$186_{10}$ & $10111010_2$ & 10111 & 101 & M \\
	\hline
	$253_{10}$ & $11111101_2$ & 11111 & 110 & M \\
	\hline
	\end{tabular}\]
	\[\begin{tabular} {|c|c|c|}
	\hline
	Block number & Word Addresses & Tag \\
	\hline
	$126_{10}$ & $252_{10},253_{10}$ & 11111 \\ 
	\hline
	$93_{10}$ & $186_{10},187_{10}$ & 10111 \\ 
	\hline
	$22_{10}$ & $44_{10},45_{10}$ & 00101 \\ 
	\hline
	$90_{10}$ & $180_{10},181_{10}$ & 10110 \\ 
	\hline
	$7_{10}$ & $14_{10},15_{10}$ & 00001 \\ 
	\hline
	$95_{10}$ & $190_{10},191_{10}$ & 10111 \\ 
	\hline
	$44_{10}$ & $88_{10},89_{10}$ & 01011 \\ 
	\hline
	$1_{10}$ & $2_{10},3_{10}$ & 00000 \\ 
	\hline
	\end{tabular}\]
\item 5.5.1: Assume a 64 KiB direct-mapped cache with a 32-byte block. What is the miss rate for the address stream above? How is this miss rate sensitive to the size of the cache or the working set? How would you categorize the misses this workload is experienceing, based on the 3C model?
\[\boxed{\frac{1}{8}=0.125 \Rightarrow 12.5\% \text{ miss rate does not change with cache size or working set} = \text{cold misses}}\]
\item 5.5.2: Re-compute the miss rate when the cache block size is 16 bytes, 64 bytes, and 128 bytes. What kind of locality is this workload exploiting?
	\[\begin{tabular} {|c|c|}
	\hline
	\textbf{Cache block size} & \textbf{miss rate} \\ 
	\hline
	16 bytes & $\frac{1}{4}=25\%$ \\
	\hline
	64 bytes & $\frac{1}{16}=6.25\%$\\
	\hline
	128 bytes & $\frac{1}{32}=3.125\%$\\
	\hline
	\end{tabular}\]
	\boxed{\text{Spacial locality}}
\item 5.5.3: ``Prefetching'' is a technique that leverages predictable address patterns to speculatively bring in additional cache blocks when a particular cache block is accessed. Assume a two-entry stream buffer and assume that the cache latency is such that a cache block can be loaded before the computation on the previous cache block is completed. What is the miss rate for the address stream above?\\
\boxed{\text{Nearly 0\%}}
\item The minimum addressable unit of a memory system is a 32-bit word. The memory system includes a four-way set-associative cache. The block size is eight words. The \textbf{data} part of the cache is implemented using a single conventional 64 $K \times 32$ RAM chip. We will call this chip C1. The connections to C1 are shown on page 10.24 in the notes. You cannot modify C1 in any way. For each address generated by the CPU, the cache controller, using the directory part of the cache, maps that address to an address in C1.
	\begin{enumerate}[label=\Alph*)]
	\item The CPU generates the address 0x321. The access is a \textbf{hit} in the cache. Specify the address in C1 where the word is found. If more than one answer is possible for your assumed implementation of the cache controller and directory, you must specify all possible answers. Addresses must be specified in hex.
	\[\frac{64 K \text{ words}}{32 \text{ words per set}} = 2048 \text{ set frames} \Rightarrow 11 \text{ bit index}\]
	Possible answer 1: \begin{tabular}{|c|c|c|}
	\hline
	\textbf{tag} & \textbf{index} & \textbf{block offset}\\
	\hline
	18 & 11 & 3 \\
	\hline
	\end{tabular}\\
	Possible answer 2: \begin{tabular}{|c|c|c|}
	\hline
	\textbf{index} & \textbf{set offset} & \textbf{block offset}\\
	\hline
	11 & 2 & 3 \\
	\hline
	\end{tabular}\\
	Possible answer 3: \begin{tabular}{|c|c|c|}
	\hline
	\textbf{set offset} & \textbf{index} & \textbf{block offset}\\
	\hline
	2 & 11 & 3 \\
	\hline
	\end{tabular}\\
	Assume possibility 1:\begin{tabular}{|c|c|c|}
	\hline
	\textbf{tag} & \textbf{index} & \textbf{block offset}\\
	\hline
	\dots & 00001100100 & 001 \\
	\hline
	\end{tabular}\\
	Possible locations in C1:
	\begin{tabular}{|c|c|}
	\hline
	\textbf{binary} & \textbf{hex} \\
	\hline
	0000110010000001 & 0x0C81\\
	\hline
	0000110010001001 & 0x0C89\\
	\hline
	0000110010010001 & 0x0C91\\
	\hline
	0000110010011001 & 0x0C99\\
	\hline
	\end{tabular}
	\item Repeat part A for address 0x9876543.
	\begin{tabular}{|c|c|c|}
	\hline
	\textbf{tag} & \textbf{index} & \textbf{block offset}\\
	\hline
	\dots & 10010101000 & 011 \\
	\hline
	\end{tabular}\\
	Possible locations in C1:
	\begin{tabular}{|c|c|}
	\hline
	\textbf{binary} & \textbf{hex} \\
	\hline
	1001010100000011 & 0x9503\\
	\hline
	1001010100001011 & 0x950B\\
	\hline
	1001010100010011 & 0x9513\\
	\hline
	1001010100011011 & 0x951B\\
	\hline
	\end{tabular}
	\end{enumerate}
\item Consider the multicycle MIPS implementation described in Figure 5.28 and 5.37 in Chapter 5, 3rd Ed. This processor executes a program will the following instruction mix:
	\begin{lstlisting}
	loads		25%
	stores		15%
	R-format	40%
	branches	18%
	jumps		2%
	\end{lstlisting}
Unlike Figure 5.28, the memory subsystem of the system you are considering consists of a \textbf{write-back, write-around} unified instruction/address cache and main memory. The access time of the cache is 1 cycle. The access time to main memory is 5 cycles. The block size is 8 bytes (2 words). The main memory word size is 32-bits. On average, 25\% of the blocks are dirty. There is no write buffer. State every assumption and be sure that it is clear how you arrive at your answer.
	\begin{enumerate}[label=\Alph*)]
	\item Compute the CPI of the system assuming the cache hit rate is 100\%.
	\[\text{CPI} = 0.25(5) + 0.15(4) + 0.40(4) + 0.18(3)+ 0.02(3) = \boxed{4.05}\]
	\item For what cache hit rate will performance decrease (i.e., execution time increase) by a factor of 2.\\
	Assume the cache can hold both data and instructions.
	\[\frac{\text{CPI}_{\text{miss}}}{\text{CPI}_{\text{100\% hit}}}=2\]
	\[\text{time}_{\text{copy block}} = 5 \text{ cycles/word} \times 2 \text{ words/block} = 10 \text{ cycles/block}\]
	75\% of misses have a clean victim block with 10 cycles to read the block from memory.
	25\% of misses have a dirty victim block with 10 cycles to write the victim block to memory and 10 cycles to read the block from memory.
	\[\text{CPI}_{\text{I-miss}}=(1-\text{cache}_{\text{hit}})\times(0.65 \times 10 + 0.35 \times 20) = 13.5\times(1-\text{cache}_{\text{hit}})\]
	\[\text{CPI}_{\text{load-miss}}=0.25\times(1-\text{cache}_{\text{hit}})\times(0.65 \times 10 + 0.35 \times 20) = 3.375\times(1-\text{cache}_{\text{hit}})\]
	\[\text{CPI}_{\text{store-miss}}=0.15\times(1-\text{cache}_{\text{hit}})\times5=0.75\times(1-\text{cache}_{\text{hit}})\]
	\[\text{CPI}_{\text{miss}}=4.05 + 13.5\times(1-\text{cache}_{\text{hit}}) + 3.375\times(1-\text{cache}_{\text{hit}}) + 0.75\times(1-\text{cache}_{\text{hit}})=21.675-17.625\times\text{cache}_{\text{hit}}\]
	\[\frac{\text{CPI}_{\text{miss}}}{\text{CPI}_{\text{100\% hit}}}=\frac{21.675-17.625\times\text{cache}_{\text{hit}}}{4.05}=2\]
	\[\text{cache}_{\text{hit}}=0.7702\Rightarrow\boxed{77\%}\]
	\end{enumerate}
\end{enumerate}
\end{document}
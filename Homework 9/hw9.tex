%%%%%%%%%% DOCUMENT STUFF %%%%%%%%%%

\documentclass[10pt,letterpaper]{article}
\usepackage{mathtools}
\usepackage{amsmath}
\usepackage{amssymb}
\usepackage{datetime}
\usepackage{setspace}
\usepackage{tkz-graph}
\usepackage[margin=1in]{geometry}
\usepackage{courier}
\usepackage{listings}
\usepackage{mips}
\usepackage{graphicx}
\usepackage{graphics}
\usepackage{enumitem}
\usepackage{colortbl}

%%%%%%%%%% FORMATTING %%%%%%%%%%

\newdate{date}{07}{06}{2017}
\spacing{1.5}
\date{\displaydate{date}}
\setcounter{secnumdepth}{0}
\newcommand\tab[1][0.5cm]{\hspace*{#1}}
\newcommand*\circled[1]{\tikz[baseline=(char.base)]{
            \node[shape=circle,draw,inner sep=2pt] (char) {#1};}}
\lstset{language=[mips]Assembler,basicstyle=\ttfamily}
\graphicspath{{images/}}

%%%%%%%%%% CONTENT %%%%%%%%%%

%%%%% COVER PAGE %%%%%

\begin{document}
\title{CS M151B: Homework 9}
\author{
	Jonathan Woong\\
	804205763\\
	Spring 2017\\
	Discussion 1A}
\maketitle
\pagebreak

%%%%%%%%%%

\begin{enumerate}[label=\textbf{Problem \arabic*.}]
\item You are a member of the design team of a new computer system. Your task is to nail down the details regarding the implementation of the cache. Some decisions cannot be changed:
	\begin{itemize}
	\item The system is designed for programs where 65\% of memory accesses are reads and 35\% of memory accesses are writes.
	\item The block size is 16 bytes.
	\item The width of the bus between the cache and main memory is 32 bits.
	\item The cache access time is 5 ns.
	\item The main memory access time is 68 ns.
	\item The write policy is write-back.
	\end{itemize}
On a write miss, there are two options:
	\begin{enumerate}[label=(\Roman*)]
	\item The main memory is updated but the accessed block is not copied to the cache (write-around). The hit rate is 90\% and on average 70\% of the blocks in the cache are dirty. 
	\item The block is moved to the cache and is updated there (write-allocate). The hit rate is 95\% and on average 75\% of the blocks in the cache are dirty.
	\end{enumerate}
In order to choose between options (I) and (II), you must compute the effective access time for each case and pick the design that will result in a lower access time. Explain any assumption you need to make. Explain briefly the computations you make, i.e., what is the justification for the way you are computing the effective access time. Check your calculations carefully — a correct decision requires correct computations.
	\begin{enumerate}[label=(\Roman*)]
	\item Assumption: 1 memory access to update main memory. $\frac{16 \text{ bytes}}{32 \text{ bits}}=4$ memory accesses to fill 1 cache block.
		\[\begin{tabular} {|c|c|c|}
		\hline
		\textbf{case} & \textbf{access time (ns)} & \textbf{probability} \\
		\hline
		read hit & 5 & $0.65\times0.9 = 0.585$ \\
		\hline
		clean read miss & $5 + (4 \times 68) = 277$ & $0.65 \times 0.1 \times 0.3 = 0.0195$ \\
		\hline
		dirty read miss & $5 + (8 \times 68) = 549$ & $0.65 \times 0.1 \times 0.7 = 0.0455$\\
		\hline
		write hit & 5 & $0.35 \times 0.9 = 0.315$ \\
		\hline
		write miss & 5 + 68 = 73 & $0.35 \times 0.1 = 0.035$ \\
		\hline
		\end{tabular}\]
		\[\text{Effective access time}_I=(5\times0.585)+(277\times0.0195)+(549\times0.0455)+(5\times0.315)+(73\times0.035)=37.436 \text{ ns}\]
	\item Load block into cache for store miss and load miss.
	\[\begin{tabular} {|c|c|c|}
		\hline
		\textbf{case} & \textbf{access time (ns)} & \textbf{probability} \\
		\hline
		cache hit & 5 & 0.95 \\
		\hline
		clean cache miss & $5 + (4 \times 68) = 277$ & $0.05 \times 0.25 = 0.0125$ \\
		\hline
		dirty cache miss & $5 + (8 \times 68) = 549$ & $0.05 \times 0.75 = 0.0375$\\
		\hline
		\end{tabular}\]
		\[\text{Effective access time}_{II}=(5\times0.95)+(277\times0.0125)+(549\times0.0375)= 28.8\text{ ns}\]
	\end{enumerate}
\boxed{\text{Choose option (II), which has a lower effective access time of 28.8 ns.}}
\item 5.11.4: 
\[\begin{tabular}{|c|c|c|}
\hline
\textbf{VA size} & \textbf{page size} & \textbf{PTE size} \\
\hline
32 bits & 8 KiB & 4 bytes \\
\hline
\end{tabular}\]
Given the parameters shown above, calculate the total page table size for a system running 5 applications that utilize half of the memory available.
\[8 \text{ KiB} = 2^{13} \text{ bytes} \Rightarrow 13 \text{ bit page offset}, 19 \text{ bit page number}\]
\[2^{19} = 0.5 \text{ M PTEs}\]
\[0.5 \text{ M PTEs} \times 4 \text{ bytes/entry} = 2 \text{ MB per page table per application}\]
\[5 \text{ applications} \times 2 \text{ MB} = \boxed{10 \text{ MB for all application page tables}}\]
\item 5.11.5: Given the parameters shown above, calculate the total page table size for a system running 5 applications that utilize half of the memory available, given a two level page table approach with 256 entries. Assume each entry of the main page table is 6 bytes. Calculate the minimum (I) and maximum (II) amount of memory required.
\[8 \text{ KiB} = 2^{13} \text{ bytes} \Rightarrow 13 \text{ bit page offset}, 19 \text{ bit page number}\]
\[8 \text{ MSB index into first-level} \Rightarrow 11 \text{ bit index into second-level} \Rightarrow 2^{11} = 2048 \text{ entries}\]
	\begin{enumerate}[label=(\Roman*)]
	\item 128 first-level entries per application
	\[128 \text{ first-level entries} \times 2048 \text{ entries per second-level page table} = 262 \text{ K entries}\]
	\[262 \text{ K entries} \times 4 \text{ bytes/entry} = 1048 \text{ KB for all second-level page tables per application}\]
	\[256 \text{ entries} \times 6 \text{ bytes} = 1536 \text{ bytes for first-level per application}\]
	\[5 \text{ applications} \times (1048 \text{ KB} + 1538 \text{ bytes}) = \boxed{5.12 \text{ MB for all page tables}}\]
	\item 256 first-level entries per application
	\[256 \text{ first-level entries} \times 2048 \text{ entries per second-level page table} = 0.5 \text{ M entries}\]
	\[0.5 \text{ M entries} \times 4 \text{ bytes/entry} = 2 \text{ MB for all second-level page tables per application}\]
	\[256 \text{ entries} \times 6 \text{ bytes} = 1536 \text{ bytes for first-level per application}\]
	\[5 \text{ applications} \times (2 \text{ MB} + 1538 \text{ bytes}) = \boxed{10.01 \text{ MB for all page tables}}\]
	\end{enumerate}
\item A virtual memory has a page size of 4096 ($2^{12}$) words. There are 64 pages of virtual address space but only 5 page frames in real memory. The page table is stored in a special memory module and does not take up space in real memory. The program issues the following sequence of addresses (hex):
\[\texttt{2f012, d120, 550, d58b, 7194, 30000, 7052, 5550, 2fa02, 744, 7276}\]
Before the program begins execution, the real memory is empty.
	\begin{enumerate}[label=\Alph*)]
	\item Assume that FIFO replacement is used. Circle the references that will cause a page fault.
	\[\texttt{\circled{2f012}, \circled{d120}, \circled{550}, d58b, \circled{7194}, \circled{30000}, 7052, \circled{5550}, \circled{2fa02}, 744, 7276}\]
	\item Assume that LRU replacement is used. Circle the references that will cause a page fault.
	\[\texttt{\circled{2f012}, \circled{d120}, \circled{550}, d58b, \circled{7194}, \circled{30000}, 7052, \circled{5550}, \circled{2fa02}, \circled{744}, 7276}\]
	\end{enumerate}
\item A computer system has the following characteristics:
	\begin{itemize}
	\item The memory is byte addressable.
	\item The disk address where a virtual page is stored is the virtual page number.
	\item Processor: 30 bit addresses. 65\% of memory accesses are reads. 
	\item Real memory: Size: 128 MiB. Access time: 110 ns. Width of bus to memory is 64 bits. Page size: 16 KiB.
	\item Cache: Accessed using physical addresses. Size: 512 KiB. Access time: 15 ns. Block size: 64 bytes. Organization: 8-way set-associative. Hit rate: 95\%. Write policy: write-back, write-around.
	\item TLB: Size: 256 entries. Access time: 5 ns. Organization: direct mapped. Hit rate: 98\%.
	\end{itemize}
	\begin{enumerate}[label=\Alph*)]
	\item Assume that the page fault rate is 0.1\% and the average disk access time is 8 ms. Specify what additional information is needed.\[\fbox{\parbox{\linewidth}{\begin{itemize}
		\item Time to (handle page fault): save current state, transfer control to page fault handling routine, initiate disk access, read from disk, detect operation completion, restart process, transfer page from disk to memory.
		\item Amount of dirty pages in main memory.
		\item Amount of dirty cache blocks.
		\item Levels of page table.
		\end{itemize}}}\]
	\item Assume that the page fault rate is 0. Compute the effective access time to the memory system. Clearly specify any assumptions you make.\\
	Let $d = $ fraction of dirty cache blocks.\\
	Let $k = $ number of levels of page table.\\
	\[\text{translation time} = 5 \text{ ns} + (0.02\times110 \text{ ns}\times k) = (5 + 2.2\times k) \text{ ns}\]
	\[\begin{tabular} {|c|c|c|}
		\hline
		\textbf{case} & \textbf{access time (ns)} & \textbf{probability} \\
		\hline
		read hit & 15 & $0.65\times0.95 = 0.6175$ \\
		\hline
		clean read miss & $15 + (8 \times 110) = 895$ & $0.65 \times 0.05 \times (1-d)$ \\
		\hline
		dirty read miss & $15 + (16 \times 110) = 1775$ & $0.65 \times 0.05 \times d$\\
		\hline
		write hit & 15 & $0.35 \times 0.95 = 0.3325$ \\
		\hline
		write miss & 15 + 110 = 125 & $0.35 \times 0.05 = 0.0175$ \\
		\hline
	\end{tabular}\]
	\begin{equation*}\hspace*{-4cm}\begin{split}
	& \text{Effective access time} =
	\text{translation time + memory access time} \\
	& = (5 + 2.2\times k) + (15 \times 0.6175) + (895 \times 0.65 \times 0.05 \times (1-d)) + (1775 \times 0.65 \times 0.05 \times d) + (15 \times 0.3325) + (125 \times 0.0175) \\
	& = \boxed{(50.525 + 2.2k + 28.6d) \text{ ns}}
	\end{split}\end{equation*}
	\end{enumerate}
\item Consider a byte-addressable virtual memory system with the following properties:
	\begin{itemize}
	\item 64 bit virtual addresses
	\item 8 KiB pages
	\item 38 bit physical address
	\item 8-way set-associative TLB with 512 page-table entries
	\item Four level page table
	\end{itemize}
Computer the total amount of storage required for the TLB. Your answer should be the total number of bits required for all storage needed for the TLB. State every assumption you make and explain the calculation.\\
8 KiB page = $2^{13} B \Rightarrow$ 13 bit page offset\\
64 - 13 = 51 bit virtual page number\\
$\frac{1024 \text{ entries}}{8 \text{ entries per frame}} = 128 = 2^7$ set frames $\Rightarrow 7$ bit index field\\
51 - 7 = 44 bit tag field\\
Virtual address\[\begin{tabular} {|c|c|c|}
\hline
44 & 7 & 13 \\
\hline
\multicolumn{2}{|c|}{VPN} & page offset \\
\hline
tag & index & page offset \\
\hline
\end{tabular}\]
38 - 13 = 25 bit PPN\\
Physical address\[\begin{tabular}{|c|c|}
\hline
25 & 13 \\
\hline
PPN & page offset \\
\hline
\end{tabular}\]
Assume TLB includes:	
	\begin{itemize}
	\item 1 valid bit
	\item 1 dirty bit
	\item 3 protection bits
	\end{itemize}
One entry = 1 + 1 + 3 + 44 + 25 = 74 bits\\
1024 entries = $(74 \times 1024) =$ \boxed{75776 \text{ bits}}
\item Consider a shared-memory multiprocessor where each processor has a local cache. The cache write policy is write-back, write-allocate. Each cache block contains two words. Elements \texttt{X[0]} and \texttt{X[1]} of array \texttt{X} are in the same block. Initially, \texttt{X[0]=X[1]=0}. Then, the two processes simultaneously execute the following C code:
\[\begin{tabular}{|c|c|}
\hline
\textbf{P1} & \textbf{P2} \\
\hline
\texttt{X[0]++; print X[1];} & \texttt{X[1] += 2; print X[0];} \\
\hline
\end{tabular}\]
	\begin{enumerate}[label=\Alph*)]
	\item There are several possible results from the execution of this code. Assume that cache coherence is maintained. List all the possible values that may be printed by this program. Your answer must be a list of pairs, where the first element of each pair is the value printed by P1 and the second element of each pair is the value printed by P2. Explain your answers.
		\[\begin{tabular}{|c|c|c|}
		\hline
		\textbf{step} & \textbf{P1} & \textbf{P2} \\
		\hline
		1 & \$1 $\leftarrow$ X[0] & \$1 $\leftarrow$ X[1] \\
		\hline
		2 & \$1 $\leftarrow$ \$1 + 1 & \$1 $\leftarrow$ \$1 + 2 \\
		\hline
		3 & X[0] $\leftarrow$ \$1 & X[1] $\leftarrow$ \$1 \\
		\hline
		4 & \$1 $\leftarrow$ X[1] & \$1 $\leftarrow$ X[0] \\
		\hline
		5 & jal print & jal print \\
		\hline
		\end{tabular}\]
		\[\begin{tabular}{|c|c|}
		\hline
		\textbf{result} & \textbf{case} \\
		\hline
		X[0] = 0 & P2$_4$ before P1$_3$ \\
		\hline
		X[0] = 1 & P1$_3$ before P2$_4$ \\
		\hline
		X[1] = 0 & P1$_4$ before P2$_3$ \\
		\hline
		X[1] = 2 & P2$_3$ before P1$_4$ \\
		\hline
		(X[0] = 0 \&\& X[1] = 0) impossible & P2$_4$ before P1$_3$ \\
		\hline
		\end{tabular}\]
	Outcomes: \boxed{(0,1),(2,0),(2,1)}
	\item Assume that cache coherence is not maintained. List at least one possible value of the cache block that is an outcome of executing of the code above and is not on the list that you produced in your answer to part A. As with part A, the answer here is at least one pair. Explain your answer.
	\[\begin{tabular}{|c|c|c|c|c|}
	\hline
	\textbf{step} & \textbf{P1} & \textbf{cache 1} & \textbf{P2} & \textbf{cache 2} \\
	\hline
	P1$_1$ & \$1 $\leftarrow$ X[0] & miss, X[0]=0, X[1]=0 & & \\
	\hline
	P2$_1$ & & & \$1 $\leftarrow$ X[1] & miss, X[0]=0, X[1]=0 \\
	\hline
	P2$_2$ & & & \$1 $\leftarrow$ \$1 + 2 & \\
	\hline
	P2$_3$ & & & X[1] $\leftarrow$ \$1 & hit, X[0]=0, X[1]=2 \\
	\hline
	P2$_4$ & & & \$1 $\leftarrow$ X[0] & hit, X[0]=0, X[1]=2 \\
	\hline
	P2$_5$ & & & jal print & \\
	\hline
	P1$_2$ & \$1 $\leftarrow$ \$1 + 1 & & & \\
	\hline
	P1$_3$ & X[0] $\leftarrow$ \$1 & hit, X[0]=1, X[1]=0 & & \\
	\hline
	P1$_4$ & \$1 $\leftarrow$ X[1] & hit, X[0]=1, X[1]=0 & & \\
	\hline
	P1$_5$ & jal print & & & \\
	\hline
	\end{tabular}\]
	\[\begin{tabular}{|c|c|}
		\hline
		\textbf{result} & \textbf{case} \\
		\hline
		X[1] = 2 & P2$_3$ \\
		\hline
		X[1] = 0 & P1$_4$ \\
		\hline
		\end{tabular}\]
	Result: \boxed{(0,0)}
	\item Assume that cache coherence is maintained using a simple write-invalidate snoopy protocol. Before execution begins, all the block frames in the caches of both processors are invalid. What is the minimum number of cache misses, in both caches, that will occur during the execution of the program. Explain your answer.\\
	\fbox{\parbox{\linewidth}{If one processor executes all of its code before the other, then we get the minimum number of cache misses. This means that the only misses are compulsory misses, making the minimum number of misses 2.}}
	\item Repeat part C but now determine the maximum number of cache misses, in both caches, that will occur during the execution of the program. Explain your answer.
	\[\hspace*{-3cm}\begin{tabular}{|c|c|c|c|c|}
	\hline
	\textbf{step} & \textbf{P1} & \textbf{cache 1} & \textbf{P2} & \textbf{cache 2} \\
	\hline
	P1$_1$ & \$1 $\leftarrow$ X[0] & miss, X[0]=0, X[1]=0 & & \\
	\hline
	P2$_1$ & & & \$1 $\leftarrow$ X[1] & miss, X[0]=0, X[1]=0 \\
	\hline
	P2$_2$ & & & \$1 $\leftarrow$ \$1 + 2 & \\
	\hline
	P2$_3$ & & invalidate (X[0],X[1]) & X[1] $\leftarrow$ \$1 & hit, X[0]=0, X[1]=2, mark as dirty \\
	\hline
	P1$_2$ & \$1 $\leftarrow$ \$1 + 1 & & & \\
	\hline
	P1$_3$ & X[0] $\leftarrow$ \$1 & miss, X[0]=1, X[1]=2, mark as dirty & & write-back \&\& invalidate (X[0]=0, X[1]=2)\\
	\hline
	P2$_4$ & & write-back X[0]=1, X[1]=2 & \$1 $\leftarrow$ X[0] & miss, X[0]=1, X[1]=2 \\
	\hline
	P2$_5$ & & & jal print & \\
	\hline
	P1$_4$ & \$1 $\leftarrow$ X[1] & hit, X[0]=1, X[1]=0 & & \\
	\hline
	P1$_5$ & jal print & & & \\
	\hline
	\end{tabular}\]
	\end{enumerate}
\boxed{\text{Maximum misses} = 4}
\end{enumerate}
\end{document}